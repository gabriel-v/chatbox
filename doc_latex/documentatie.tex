\documentclass[14pt,a4paper]{extarticle}
\usepackage[utf8x]{inputenc}
\usepackage[romanian]{babel}
\usepackage{lmodern}
\usepackage[pdftex]{graphicx}
\usepackage[super,comma]{natbib}
\usepackage{indentfirst}
\usepackage{paralist}
  \let\itemize\compactitem
  \let\enditemize\endcompactitem
  \let\enumerate\compactenum
  \let\endenumerate\endcompactenum
  \let\description\compactdesc
  \let\enddescription\endcompactdesc
  \pltopsep=\medskipamount
  \plitemsep=1pt
  \plparsep=1pt
\newcommand{\HRule}{\rule{\linewidth}{0.5mm}}

\renewcommand*\contentsname{Cuprins}

\begin{document}

\begin{titlepage}
\begin{center}

\textsc{\LARGE Colegiul National \\[0.5cm] "Mihai Viteazul"}\\[1.5cm]

\textsc{\Large Atestat la Informatica}\\[0.5cm]

% Title
\HRule \\[0.4cm]
{ \huge \bfseries chatbox \\[0.4cm] }

\HRule \\[1.5cm]

% Author and supervisor
\noindent
\begin{minipage}[t]{0.4\textwidth}
\begin{flushleft} \large
\emph{Elev:}\\
\textsc{Vijiala\\Tudor-Gabriel}
\end{flushleft}
\end{minipage}%
\begin{minipage}[t]{0.4\textwidth}
\begin{flushright} \large
\emph{Profesor indrumator:} \\
\textsc{Stan\\ Mihaela-Veronica}
\end{flushright}
\end{minipage}

\vfill

% Bottom of the page
{\large 18 mai, 2015}

\end{center}
\end{titlepage}

\newpage
\section{Introducere}

\newpage
\section{Resurse logice}
Resursele logice sunt componentele software ale calculatorului, care au functii de administrare a resurselor si a datelor.

In cazul proiectului \textbf{chatbox}, acestea reprezinta mijlocul prin care
pagina web este programata, monitorizata si administrata.

\subsection{Sistemul MySQL}
MySQL\cite{mysql} este cel mai folosit SGBD open-source, 
la ora actuala. Produs initial de compania suedeza MySQL AB 
și distribuit sub Licența Publică Generală GNU\cite{free}, 
in prezent MySQL este dezvoltat
de Corporatia Oracle.

Ca instrument de management pentru bazele de date MySQL
este folosita o aplicatie PHP numita phpMyAdmin.

\subsection{Limbajul PHP}
Limbajul de programare PHP\citep{php} este folosit pe scară largă 
în dezvoltarea paginilor și aplicațiilor web.

Se folosește în principal înglobat în codul HTML, dar poate fi 
utilizat si pentru programarea aplicatiilor CLI (linie de comanda).

PHP este disponibil sub Licenṭa PHP ṣi Free Software Foundation 
îl consideră a fi un software liber\citep{free}.
\subsubsection{Libraria PDO}
PHP Data Objects\citep{pdo} este o componenta PHP ce permite accesarea unor SGBD din
programe PHP. \textbf{chatbox} foloseste in mod exclusiv componenta PDO pentru 
accesarea bazei de date. 
\subsubsection{Libraria Ratchet}
Websocket\citep{websocket} este un protocol ce furnizeaza o conexiune duplex prin o legatura TCP. Tehnologia Websocket a fost dezvoltata odata cu initiativa de 
inovare HTML5. 

Folosind tehnologia Websocket, se pot trimite date in timp real intre 
client si server. \textbf{chatbox} foloseste pentru transmiterea instantanee
a mesajelor acest protocol.

Ratchet\citep{ratchet} este o librarie ce permite utilizarea protocolului 
Websocket, in limbajul PHP.
\subsection{Limbajul JavaScript}
\subsubsection{Libraria jQuery}
\subsubsection{Libraria chart.js}
\subsection{Limbajul Python}

\subsection{Despre serverul Linux}
Codul proiectului \textbf{chatbox} calculator personal
 \textit{HP Compaq 6005 Pro}
care serveste, printre altele, drept server web. 

Pentru a dispune de o adresa permanenta a acestui server, 
am folosit serviciile de DNS dinamic ale firmei NoIP\cite{noip}.

Sistemul are urmatoarele caracteristici:
\begin{itemize}
  \item Procesor: AMD Athlon II X2 2.8GHz
  \item Memorie: 2GB RAM DDR3
  \item Hard Drive: 2TB Samsung
  \item Placa retea: 100Mbps
\end{itemize}

\subsubsection{Sistemul de operare: Debian 8}
Debian\cite{debian} este un sistem de operare compus din software liber\cite{free}, și o distribuție populară și foarte influentă între distribuțiile GNU/Linux.

Versiunea 8 a sistemului de operare este cunoscuta in prezent sub 
numele de \textit{testing}. Pachetele de software pentru versiunea 
\textit{testing} sunt, dupa cum sugereaza si numele, in curs de testare.
Totusi, sunt destul de stabile pentru modul in care 
utilizez acest sistem.
\subsubsection{Serverul HTTP: Apache 2}
Apache este un server HTTP open-source. Acesta reprezinta standardul in 
industria de web hosting, fiind cel mai folosit server HTTP, fiind folosit de 
53.34\% din site-urile web\cite{apache53}.
\subsubsection{Sistemul de daemon: daemontools}
Daemontools\cite{daemon} este o colectie de unelte software folosite pentru 
controlul si monitorizarea serviciilor UNIX. 

Daemontools este folosit de \textbf{chatbox} pentru a monitoriza serverul 
de \textit{websockets} si a-l reporni in eventualitatea unei erori.
\newpage
\section{Descrierea proiectului}

\subsection{Baza de date}

\subsection{Sistemul de autentificare}

\subsection{Sistemul de mesagerie}

\subsubsection{Interfata}

\subsubsection{Serverul de websockets}

\subsection{Generarea si vizualizarea datelor}

\subsubsection{Generatorul de date}

\subsubsection{Vizualizarea datelor prin grafice}

\newpage
\section{Bibliografie}
\begingroup
\renewcommand{\section}[2]{}%
\begin{thebibliography}{99}

\bibitem{mysql} 
	MySQL \\
	http://www.mysql.com/about/

\bibitem{php}
	PHP Hypertext Processor\\
	http://php.net/manual/en/intro-whatis.php

\bibitem{pdo}
	PDO: PHP Data Objects\\
	http://php.net/manual/en/intro.pdo.php
	
\bibitem{websocket}
	Websocket\\
	https://www.websocket.org/
	
\bibitem{ratchet}
	Ratchet: a Websockets Library\\
	http://socketo.me/
	
\bibitem{javascript}
	Objects in Javascript\\
	http://www.w3.org/community/webed/wiki/Objects\_in\_JavaScript
	
\bibitem{jquery}
	jQuery: The Write Less, Do More, JavaScript Library\\
	https://jquery.com/

\bibitem{chartjs}
	Chart.js: Open source HTML5 Charts \\
	http://www.chartjs.org/
	
\bibitem{python}
	Python \\
	https://www.python.org/about/
	
\bibitem{noip}
	No-IP: Dynamic DNS company \\
	http://www.noip.com/
	
\bibitem{debian}
	Debian GNU/Linux \\
	https://www.debian.org/intro/about

\bibitem{free}
	Software liber / Free Software \\
	https://www.fsf.org/about/what-is-free-software
	
\bibitem{apache53}
	June 2013 Web Server Survey\\
http://news.netcraft.com/archives/2013/06/06/june-2013-web-server-survey-3.html
	
\bibitem{daemon}
	daemontools\\
	http://cr.yp.to/daemontools.html

\end{thebibliography}
\endgroup
\newpage
\tableofcontents

\end{document}